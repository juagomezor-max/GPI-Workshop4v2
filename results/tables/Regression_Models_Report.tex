% Regression Modeling Results
% Generated automatically by 05_modelado.R

\documentclass[11pt,a4paper]{article}
\usepackage[utf-8]{inputenc}
\usepackage[T1]{fontenc}
\usepackage[spanish]{babel}
\usepackage{booktabs}
\usepackage{float}
\usepackage{longtable}
\usepackage{geometry}
\usepackage{amsmath}

\geometry{margin=1in}

\title{Statistical Regression Modeling Report}
\author{GPI Workshop 4}
\date{\today}

\begin{document}
\maketitle

\tableofcontents
\newpage

\section{Introduction}

This document presents a comprehensive statistical regression analysis with multiple models, 
diagnostics, and comparisons.

\section{Model Specifications and Comparison}

Eight different regression models were fitted to the data:

\begin{enumerate}
  \item Simple regressions: y ~ x, z ~ x, z ~ y
  \item Multiple regressions: z ~ x + y, y ~ x + grupo, z ~ x + y + grupo
  \item Interaction models: z ~ x * y, y ~ x * grupo
\end{enumerate}

Table \ref{tab:model-comparison} presents the comparison metrics for all models.

\begin{table}[H]
  \centering
  \caption{Model Comparison Metrics}
  \label{tab:model-comparison}
  
\begin{longtable}{llrrrrrrr}
\toprule
  & Model & R\_squared & Adj\_R\_squared & AIC & BIC & RMSE & F\_statistic & p\_value\\
\midrule
y \textasciitilde{} x & y \textasciitilde{} x & 0.5689 & 0.5668 & 1644.762 & 1654.657 & 14.5555 & 261.3341 & 0.0000\\
z \textasciitilde{} x & z \textasciitilde{} x & 0.0064 & 0.0014 & 1785.992 & 1795.887 & 20.7188 & 1.2763 & 0.2600\\
z \textasciitilde{} y & z \textasciitilde{} y & 0.0027 & -0.0023 & 1786.736 & 1796.631 & 20.7574 & 0.5358 & 0.4651\\
z \textasciitilde{} x + y & z \textasciitilde{} x + y & 0.0066 & -0.0035 & 1787.959 & 1801.152 & 20.7171 & 0.6513 & 0.5225\\
y \textasciitilde{} x + grupo & y \textasciitilde{} x + grupo & 0.5752 & 0.5687 & 1645.834 & 1662.325 & 14.4493 & 88.4665 & 0.0000\\
\addlinespace
z \textasciitilde{} x + y + grupo & z \textasciitilde{} x + y + grupo & 0.0334 & 0.0136 & 1786.485 & 1806.275 & 20.4355 & 1.6839 & 0.1552\\
z \textasciitilde{} x * y & z \textasciitilde{} x * y & 0.0066 & -0.0086 & 1789.957 & 1806.448 & 20.7170 & 0.4326 & 0.7299\\
y \textasciitilde{} x * grupo & y \textasciitilde{} x * grupo & 0.5840 & 0.5732 & 1645.664 & 1668.753 & 14.2995 & 54.4625 & 0.0000\\
\bottomrule
\end{longtable}

\end{table}

\section{Regression Coefficients}

Table \ref{tab:coefficients} shows the detailed coefficients for all models.

\begin{table}[H]
  \centering
  \caption{Regression Coefficients for All Models}
  \label{tab:coefficients}
  
\begin{longtable}{llrrrl}
\toprule
Model & Term & Estimate & Std\_Error & t\_statistic & Significant\\
\midrule
y \textasciitilde{} x & (Intercept) & 5.676475 & 5.864747 & 0.9679 & ns\\
y \textasciitilde{} x & x & 1.882148 & 0.116428 & 16.1658 & ***\\
y \textasciitilde{} x * grupo & (Intercept) & 3.924438 & 10.572914 & 0.3712 & ns\\
y \textasciitilde{} x * grupo & x & 1.951115 & 0.207221 & 9.4156 & ***\\
y \textasciitilde{} x * grupo & as.factor(grupo)B & -8.442436 & 14.656071 & -0.5760 & ns\\
\addlinespace
y \textasciitilde{} x * grupo & as.factor(grupo)C & 15.367161 & 14.347331 & 1.0711 & ns\\
y \textasciitilde{} x * grupo & x:as.factor(grupo)B & 0.147523 & 0.287560 & 0.5130 & ns\\
y \textasciitilde{} x * grupo & x:as.factor(grupo)C & -0.400699 & 0.286146 & -1.4003 & ns\\
y \textasciitilde{} x + grupo & (Intercept) & 8.388794 & 6.137959 & 1.3667 & ns\\
y \textasciitilde{} x + grupo & x & 1.862359 & 0.116765 & 15.9496 & ***\\
\addlinespace
y \textasciitilde{} x + grupo & as.factor(grupo)B & -1.063670 & 2.521888 & -0.4218 & ns\\
y \textasciitilde{} x + grupo & as.factor(grupo)C & -4.166080 & 2.542013 & -1.6389 & ns\\
z \textasciitilde{} x & (Intercept) & 111.876612 & 8.348103 & 13.4014 & ***\\
z \textasciitilde{} x & x & -0.187228 & 0.165727 & -1.1297 & ns\\
z \textasciitilde{} x * y & (Intercept) & 110.474327 & 33.256392 & 3.3219 & **\\
\addlinespace
z \textasciitilde{} x * y & x & -0.194808 & 0.712578 & -0.2734 & ns\\
z \textasciitilde{} x * y & y & 0.032408 & 0.363980 & 0.0890 & ns\\
z \textasciitilde{} x * y & x:y & -0.000283 & 0.007008 & -0.0404 & ns\\
z \textasciitilde{} x + y & (Intercept) & 111.772707 & 8.388346 & 13.3248 & ***\\
z \textasciitilde{} x + y & x & -0.221679 & 0.253040 & -0.8761 & ns\\
\addlinespace
z \textasciitilde{} x + y & y & 0.018304 & 0.101407 & 0.1805 & ns\\
z \textasciitilde{} x + y + grupo & (Intercept) & 105.760456 & 8.744456 & 12.0946 & ***\\
z \textasciitilde{} x + y + grupo & x & -0.237549 & 0.250974 & -0.9465 & ns\\
z \textasciitilde{} x + y + grupo & y & 0.046005 & 0.101279 & 0.4542 & ns\\
z \textasciitilde{} x + y + grupo & as.factor(grupo)B & 3.803100 & 3.577438 & 1.0631 & ns\\
\addlinespace
z \textasciitilde{} x + y + grupo & as.factor(grupo)C & 8.432524 & 3.628962 & 2.3237 & *\\
z \textasciitilde{} y & (Intercept) & 107.415115 & 6.750518 & 15.9121 & ***\\
z \textasciitilde{} y & y & -0.048705 & 0.066540 & -0.7320 & ns\\
\bottomrule
\end{longtable}

\end{table}

\section{Model Diagnostics}

Diagnostic checks for Model 2.1 (z ~ x + y) are presented below.

\subsection{Residual Analysis}

Table \ref{tab:residual-stats} shows residual statistics.

\begin{table}[H]
  \centering
  \caption{Residual Statistics}
  \label{tab:residual-stats}
  
\begin{tabular}{lr}
\toprule
Metric & Value\\
\midrule
Mean of Residuals & 0.00000\\
Std Dev of Residuals & 20.76906\\
Min Residual & -58.73357\\
Max Residual & 65.42834\\
Durbin-Watson Statistic & 1.91800\\
\bottomrule
\end{tabular}

\end{table}

Sample of individual residual diagnostics is shown in Table \ref{tab:diagnostics}.

\begin{table}[H]
  \centering
  \caption{Diagnostic Values (Sample of First 20 Observations)}
  \label{tab:diagnostics}
  
\begin{longtable}{rrrrrrr}
\toprule
Observation & Fitted & Residuals & Standardized\_Residuals & Studentized\_Residuals & Cook\_Distance & Leverage\\
\midrule
1 & 100.1541 & 8.8506 & 0.4276 & 0.4267 & 0.0010 & 0.0167\\
2 & 102.8289 & -24.0340 & -1.1759 & -1.1770 & 0.0198 & 0.0413\\
3 & 103.9669 & -18.2641 & -0.8787 & -0.8782 & 0.0022 & 0.0085\\
4 & 103.4125 & -13.9502 & -0.6704 & -0.6695 & 0.0010 & 0.0063\\
5 & 104.8202 & 0.4006 & 0.0195 & 0.0194 & 0.0000 & 0.0311\\
\addlinespace
6 & 104.6281 & 4.5180 & 0.2200 & 0.2194 & 0.0005 & 0.0317\\
7 & 105.4408 & 5.3282 & 0.2605 & 0.2599 & 0.0009 & 0.0402\\
8 & 102.6327 & 22.1240 & 1.0710 & 1.0714 & 0.0081 & 0.0207\\
9 & 102.5531 & -8.9825 & -0.4314 & -0.4305 & 0.0003 & 0.0050\\
10 & 101.8909 & 28.8770 & 1.3931 & 1.3965 & 0.0091 & 0.0139\\
\addlinespace
11 & 100.8219 & 11.1669 & 0.5403 & 0.5393 & 0.0019 & 0.0196\\
12 & 103.3236 & -13.4904 & -0.6482 & -0.6473 & 0.0008 & 0.0060\\
13 & 100.0022 & -14.2209 & -0.6879 & -0.6870 & 0.0031 & 0.0192\\
14 & 103.1780 & -0.6489 & -0.0313 & -0.0312 & 0.0000 & 0.0140\\
15 & 100.4163 & -27.1647 & -1.3392 & -1.3419 & 0.0353 & 0.0557\\
\addlinespace
16 & 100.7072 & 12.9075 & 0.6221 & 0.6212 & 0.0016 & 0.0122\\
17 & 103.8923 & 8.3945 & 0.4043 & 0.4034 & 0.0006 & 0.0104\\
18 & 100.7679 & -14.3723 & -0.6924 & -0.6915 & 0.0018 & 0.0111\\
19 & 101.4442 & 8.7660 & 0.4216 & 0.4207 & 0.0005 & 0.0077\\
20 & 103.1778 & -2.1189 & -0.1023 & -0.1020 & 0.0001 & 0.0152\\
\bottomrule
\end{longtable}

\end{table}

\subsection{Multicollinearity Assessment}

Table \ref{tab:vif} presents Variance Inflation Factors (VIF) for assessing multicollinearity.
VIF values less than 5 indicate acceptable levels of multicollinearity.

\begin{table}[H]
  \centering
  \caption{Variance Inflation Factors (VIF)}
  \label{tab:vif}
  
\begin{tabular}{llr}
\toprule
  & Variable & VIF\\
\midrule
x & x & 2.3199\\
y & y & 2.3199\\
\bottomrule
\end{tabular}

\end{table}

\section{Key Findings}

\begin{itemize}
  \item The best fitting models are those with multiple predictors and interactions
  \item Model comparison metrics (R², AIC, BIC) guide model selection
  \item Diagnostic checks ensure assumptions are met
  \item VIF values confirm acceptable multicollinearity levels
\end{itemize}

\section{Conclusions}

Statistical regression modeling provides valuable insights into the relationships 
between variables. Model diagnostics and comparisons ensure robust and reliable results.

\end{document}
